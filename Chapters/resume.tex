\phantomsection
\addcontentsline{toc}{chapter}{Résumé}

\begin{center}
  {\Huge\bfseries Résumé}
\end{center}

\vspace{1.5em}

\begin{adjustwidth}{2em}{2em}
\begin{sloppypar}
Ce mémoire analyse le rôle de la forward guidance (politique monétaire) de la Banque Centrale Européenne dans la modération ou l’amplification du stress systémique au sein de la zone euro. Dans un contexte post-crise subprimes marqué par des taux d’intérêt à leur borne inférieure suivi d'une normalisation importante post-COVID, la communication anticipative est devenue un instrument central de la politique monétaire. Cependant, son efficacité demeure sujette à débat, notamment lorsque les marchés perçoivent des incohérences entre discours et actions, ou lorsque les chocs macro-financiers rendent les anticipations instables. Afin d’évaluer empiriquement l’impact de la FG sur le stress systémique, ce mémoire adopte une approche pluridisciplinaire mobilisant les outils de traitement automatique du langage et les modèles d’apprentissage profond. À partir d’un corpus de discours officiels de la BCE couvrant la période 2005–2024, plusieurs architectures neuronales sont déployées notamment des Transformers modernes (ModernBERT), modèles séquentiels (Mamba, Mamba-2, Llamba) afin de construire un indicateur sémantique continu de tonalité monétaire. Cet indicateur est ensuite intégré dans des modèles neuroneux temporels (autoencodeurs LSTM et WAE) pour évaluer son pouvoir explicatif sur l’indicateur composite de stress systémique (CISS) développé par la BCE selon différents horizons temporels (court, moyen et long terme). Les résultats mettent en évidence que la FG peut, dans certaines configurations, réduire les tensions sur les marchés financiers, à condition qu’elle soit perçue comme crédible, cohérente et adossée à des mesures concrètes. À l’inverse, une guidance ambiguë ou instable semble perdre de son efficacité, voire amplifier l’incertitude (notamment à long terme). Ce mémoire propose ainsi une lecture renouvelée de la FG comme vecteur de stabilisation systémique, tout en soulignant les limites de son action selon le contexte de normalisation monétaire.
\end{sloppypar}
\end{adjustwidth}

\vspace{2em}

\noindent\textbf{Mots-clés :} \textit{Forward Guidance, Banque Centrale Européenne, Stress Systémique, Deep Learning, Textmining, Politique Monétaire.}

