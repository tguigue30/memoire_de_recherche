\chapter*{Introduction}
\addcontentsline{toc}{chapter}{Introduction}

Depuis les débuts de la politique monétaire moderne, les banques centrales ont cherché à stabiliser l’activité 
économique en régulant les conditions de financement. Durant la majeure partie du XX\textsuperscript{e} siècle, cette 
régulation s’est exercée principalement par l’ajustement des taux d’intérêt directeurs, outil privilégié pour agir sur 
l’inflation, la croissance, ou encore la stabilité externe. Cette logique d’intervention reposait sur un paradigme 
relativement linéaire : en modulant le prix de la monnaie, la banque centrale influençait l’investissement, la 
consommation et, par extension, l’ensemble de l’économie réelle. C’est dans ce cadre que s’est affirmée, à partir des 
années 1990, l’indépendance croissante des banques centrales, accompagnée d’une cible explicite d’inflation, comme 
dans les cas de la Banque d’Angleterre ou de la BCE, créée en 1998 avec pour mandat premier la stabilité des prix 
(Art. 127 TFUE). Cependant, ce cadre d’analyse et d’intervention a été profondément remis en cause par les crises du 
début du XXI\textsuperscript{e} siècle. L’effondrement du système bancaire mondial en 2008, la fragmentation du marché 
obligataire européen durant la crise des dettes souveraines (2010–2012), puis l’épisode de stagnation prolongée 
accompagnée de taux durablement bas, ont mis à l’épreuve la capacité des banques centrales à préserver la stabilité 
financière avec les instruments traditionnels. Lorsque les taux d’intérêt nominaux ont atteint leur borne inférieure 
effective (\textit{zero lower bound}), les leviers classiques ont perdu en efficacité. Dans ce contexte, les autorités monétaires ont été contraintes d’innover.\\

C’est ainsi qu’est apparue, dans les années 2010, une nouvelle catégorie d’outils dits non conventionnels : programmes 
d’achats d’actifs, opérations ciblées de refinancement à long terme, et surtout, Forward Guidance. Cette dernière, 
définie comme une stratégie de communication sur l’orientation future de la politique monétaire, a pour objectif 
d’influencer les anticipations des agents économiques afin de renforcer les effets des mesures prises, même en 
l’absence d’ajustement immédiat des taux. Son efficacité repose sur un mécanisme central de la macroéconomie moderne : 
les décisions économiques présentes dépendent des anticipations futures, lesquelles peuvent être modulées par la 
crédibilité, la clarté et la constance des signaux envoyés par la banque centrale \citep{woodford2012}. 
Parallèlement à cette transformation des instruments monétaires, les préoccupations relatives à la stabilité 
financière se sont accrues. La crise des subprimes a mis en lumière l’existence d’effets de réseau, de phénomènes 
d’illiquidité et de spirales de ventes d’actifs qui, à partir d’un choc initial localisé, peuvent se propager à 
l’ensemble du système économique et menacer sa stabilité. Cette propagation rapide et non linéaire est désignée sous 
le terme de stress systémique. Celui-ci ne se réduit pas à une simple hausse de la volatilité : il exprime une rupture 
de confiance généralisée, une défaillance des canaux de transmission, et une perte de contrôle sur les anticipations 
collectives.\\

La décennie 2010-2020 a été profondément bouleversée par la crise du COVID-19. Face à l’arrêt brutal de l’activité économique et à la hausse généralisée de l’incertitude, la BCE a réactivé massivement sa boîte à outils non conventionnels : achats d’urgence (PEPP), refinancements ultra-accommodants (TLTRO III), et intensification de la forward guidance avec des engagements renforcés sur la trajectoire future des taux et du bilan. La communication de politique monétaire est alors devenue un outil de stabilisation à part entière, jouant un rôle crucial dans la gestion des anticipations des marchés. Mais l’environnement a de nouveau changé à partir de 2022. L’inflation élevée, initialement perçue comme transitoire, s’est enracinée, contraignant la BCE à entamer une phase rapide de resserrement monétaire. Dès lors, un nouveau régime s’installe, marqué par une normalisation des taux d’intérêt, une réduction progressive du bilan, et un repositionnement stratégique de la communication prospective. \citep{hofmann2024ecb} souligne avec justesse que cette transition vers la normalisation a posé des défis inédits pour la forward guidance : les messages prospectifs ont dû s’adapter à une incertitude élevée, à une asymétrie entre les risques de surchauffe et de ralentissement, et à la nécessité de restaurer la crédibilité de la cible d’inflation. Dans cette phase, la FG a évolué d’un instrument d’ancrage à un outil de signalisation conditionnelle, souvent ambiguë, parfois jugée incohérente avec les mesures effectivement prises.

\begin{quote}
\textit{« The ECB’s forward guidance was originally introduced as an instrument to provide additional stimulus when policy rates were close to their lower bound. However, in the post-pandemic context, forward guidance may have lost part of its effectiveness due to shifting macroeconomic risks, market scepticism, and policy reversals. »}
\hfill \citep{hofmann2024ecb}
\end{quote}

Cette séquence souligne un point important qu'il est nécessaire de bien comprendre dans l'environnement macro-monétaire : la FG est devenue un élément structurant de la politique monétaire contemporaine, mais dont les effets ne sont ni constants ni univoques. Son efficacité dépend du régime macro-financier dans lequel elle s’insère, des anticipations qu’elle façonne, et des réactions de marché qu’elle suscite. Dans un monde de plus en plus interconnecté, où les marchés sont instantanément réactifs aux signaux politiques, monétaires et géopolitiques, la surveillance du stress systémique est devenue une priorité pour les régulateurs.\\

Au-delà à ces transformations économiques et institutionnelles, une révolution technologique silencieuse a profondément modifié les méthodes d’analyse des textes économiques : l’émergence de l’intelligence artificielle, et en particulier du deep learning. Ce recours aux méthodes de deep learning n’est pas simplement motivé par un effet de mode technologique, mais par des considérations théoriques et empiriques. Les discours de politique monétaire, en particulier ceux de la BCE, présentent des caractéristiques linguistiques complexes : redondance stratégique, prudence lexicale, multiplicité des registres, présence de signaux implicites. Ces éléments rendent les méthodes économétriques classiques peu adaptées à leur analyse fine, notamment lorsqu’il s’agit d’identifier des tonalités latentes ou d’extraire des signaux prospectifs non linéaires. De plus, la dynamique du stress systémique est elle-même marquée par des non-linéarités et des régimes multiples qui nécessitent des modèles capables d’apprendre des structures complexes à partir des données. Les architectures neuronales modernes, en particulier les Transformers et les autoencodeurs variationnels régularisés, offrent précisément cette capacité d’apprentissage profond, en s’affranchissant des hypothèses paramétriques trop restrictives. Leur utilisation dans ce mémoire répond donc à un double besoin : celui de capter la richesse sémantique des textes économiques, et celui de modéliser les dynamiques complexes du stress financier de manière flexible et performante.\\

Longtemps réservées aux sciences du langage ou de l’image, les techniques d’apprentissage profond ont progressivement conquis la recherche en économie, en finance, et plus récemment en macroéconomie monétaire. Leur capacité à extraire des régularités complexes dans des données non structurées a ouvert de nouvelles perspectives pour l’analyse des discours des banques centrales, des procès-verbaux de réunion, ou encore des communiqués de presse de politique monétaire. Les premières applications ont reposé sur des modèles de classification supervisée ou sur des représentations vectorielles simples comme TF-IDF. Mais depuis la fin des années 2010, ce champ a été bouleversé par l’apparition d’architectures neuronales de type Transformers \citep{vaswani2017attention}, dont les performances dépassent celles des modèles récurrents traditionnels. Leurs mécanismes d’attention permettent de capturer efficacement les dépendances contextuelles dans des séquences textuelles longues, ce qui est particulièrement adapté au traitement des discours économiques, riches en subtilités sémantiques, en métaphores prudentes et en signaux implicites. Ces modèles ont été rapidement adoptés dans les banques centrales elles-mêmes : la BCE, la Fed ou encore la Banque d’Angleterre ont multiplié les travaux internes mobilisant BERT, GPT ou leurs déclinaisons pour évaluer la tonalité des discours, mesurer l’incertitude linguistique ou anticiper les réactions de marché \citep{gali2008,hansen2016}. Plus récemment, de nouvelles variantes de Transformers à mémoire longue ou à structure d’état comme Mamba ou Llamba ont émergé, permettant de mieux modéliser les dépendances temporelles dans des séries de textes économiques publiés à intervalles réguliers. Couplés à des architectures probabilistes comme les autoencodeurs variationnels ou les Wasserstein Auto-Encoders (WAE), ces outils autorisent une modélisation fine et continue de la tonalité monétaire. Cette convergence entre techniques de traitement automatique du langage, apprentissage profond et économétrie appliquée constitue aujourd’hui un champ de recherche en plein essor, dans lequel ce mémoire se propose en prolongation.\\

Donc dans un monde de plus en plus interconnecté, où les marchés sont instantanément réactifs aux signaux politiques, 
monétaires et géopolitiques, la surveillance du stress systémique est devenue une priorité pour les régulateurs avec de nouvelles méthodes qui permettent d'extraire ces signaux. Les 
banques centrales ne peuvent plus se limiter à l’atteinte d’une cible d’inflation ou à la régulation des agrégats 
monétaires. Elles doivent aussi veiller à la stabilité globale du système financier, coordonner leurs actions avec les 
autorités macroprudentielles, et intégrer dans leurs décisions les risques de contagion entre marchés. L’indicateur  composite de stress systémique (CISS), développé par la BCE \citep{hollo2012ciss}, incarne cette volonté de 
quantifier et de surveiller les tensions financières multidimensionnelles dans une logique systémique. Alors, les 
instruments monétaires s’étendent au-delà des ajustements techniques pour devenir des outils de communication 
stratégique. \textbf{L'objectif de ce mémoire est donc d'étudier l'impact de la Forward Guidance de la Banque Centrale Européenne sur le stress systémique, en particulier sa capacité à stabiliser le système économique ou amplifier les tensions en période de crise}. En d’autres termes, la  communication de la banque centrale est-elle en mesure de stabiliser les anticipations collectives et d’éviter les 
comportements mimétiques qui fragilisent le système ? Ou bien, au contraire, peut-elle devenir un facteur 
d’instabilité lorsqu’elle est perçue comme ambiguë, réversible ou incohérente avec les actions prises ? Cette 
interrogation conduit à une problématique plus générale :

\begin{quote}
\textit{Dans quelle mesure la communication anticipative de la politique monétaire européenne (en particulier à travers la forward guidance) permet-elle de réduire l’incertitude des marchés financiers et de prévenir les dynamiques de contagion systémique ? Ou, au contraire, peut-elle, lorsqu’elle est perçue comme ambiguë ou non crédible, amplifier les tensions sur les marchés interconnectés pour se propager à l'ensemble du système ?}
\end{quote}

Répondre à cette problématique suppose de quantifier le contenu sémantique des discours de la BCE, de modéliser leur impact temporel sur des indicateurs agrégés de stress, et d’identifier les conditions (forme, période, tonalité) sous lesquelles la forward guidance agit comme facteur stabilisateur ou déstabilisant. Cela nécessite l’articulation d’une approche théorique, computationnelle et empirique, que ce mémoire s’attache à développer.\\

Ainsi, ce mémoire se propose d’examiner la forward guidance non plus uniquement comme un outil de stabilisation conjoncturelle, mais comme un vecteur potentiel de réduction ou d’amplification du stress systémique au sein de la zone euro. L’hypothèse sous-jacente est que la nature, la fréquence et la cohérence des communications officielles de la BCE peuvent conditionner l’intensité des tensions sur les marchés financiers. Une guidance crédible, perçue comme cohérente et assortie d’engagements opérationnels tangibles, pourrait apaiser les anticipations et réduire les comportements de panique. À l’inverse, une guidance floue ou sujette à révision pourrait accroître l’incertitude et catalyser des dynamiques de contagion financière. Cette réflexion est d’autant plus pertinente dans le cadre européen, où l’unicité de la politique monétaire cohabite avec une diversité institutionnelle, économique et budgétaire propre à engendrer des divergences de perception et des fragilités structurelles. Ce travail de recherche entend donc apporter une contribution originale à la littérature économique et financière, en associant trois dimensions complémentaires. Pour commencer, une lecture historique et analytique de l’évolution de la politique monétaire de la BCE, et de la place grandissante qu’y occupe la forward guidance depuis la crise financière de 2008. Ensuite, une approche textuelle et computationnelle innovante, fondée sur les outils du traitement automatique du langage et de l’apprentissage profond, pour extraire un indicateur sémantique de forward guidance à partir des discours de la BCE. Enfin, une évaluation empirique de l’impact de cet indicateur sur le stress systémique, à travers des modèles neuronaux, avec une attention particulière portée à la robustesse des résultats et aux implications de politique économique. Pour ce faire, le mémoire est structuré en trois chapitres, chacun répondant à une étape clé de la problématique.\\

Le premier chapitre traite donc des fondations théoriques de l’analyse économique. Il débute par une rétrospective de la politique monétaire européenne, depuis la mise en place des taux directeurs par la BCE à la fin des années 1990, jusqu’aux évolutions récentes en contexte post-pandémique. Les instruments traditionnels sont confrontés aux limites de la borne zéro, ce qui ouvre la voie à des outils non conventionnels. La forward guidance est ensuite étudiée dans sa diversité : qualitative, basée sur des délais ou conditionnelle à des indicateurs économiques. Ses mécanismes de transmission sont détaillés à travers le prisme des taux d’intérêt futurs, des anticipations d’inflation, des conditions de financement et des signaux macroéconomiques. Enfin, le chapitre introduit la notion de stress systémique, sa conceptualisation théorique, ses canaux de propagation (effets domino, interdépendances de marché, spirales d’illiquidité), et ses instruments de mesure. L’indicateur composite de stress systémique (CISS) y tient une place centrale, car il permet de quantifier les tensions agrégées dans un cadre temporel et sectoriel.
Le deuxième chapitre est consacré à l’outillage méthodologique mobilisé pour répondre à l’objectif de recherche. Il présente les fondements des réseaux neuronaux appliqués au langage naturel, depuis les word embeddings jusqu’aux Transformers, avec un accent mis sur les variantes modernes utilisées dans ce mémoire (ModernBERT, Mamba, Mamba-2 et Llamba). Ces architectures permettent d’extraire une représentation vectorielle contextualisée des textes, propre à capter la tonalité, les nuances sémantiques et la structure logique des discours de politique monétaire. Le chapitre expose ensuite les principes des autoencodeurs variationnels et en particulier des Wasserstein Auto-Encoders (WAE), utilisés pour la régularisation de l’espace latent textuel et temporel. L’ensemble de ces techniques est mobilisé pour construire un indicateur continu de forward guidance, permettant de quantifier la tonalité monétaire (expansionniste, neutre, restrictive) des discours officiels de la BCE sur une période allant de 2005 à 2024.
Le troisième chapitre met en œuvre l’application empirique. À partir d’un corpus structuré des publications officielles de la BCE (communiqués de politique monétaire, discours présidentiels, conférences de presse), un indicateur de forward guidance est généré pour chaque date de décision. Ce score est ensuite intégré dans différents modèles prédictifs du stress systémique : régressions Ridge pénalisées pour éviter le sur-ajustement, modèles à mémoire courte (LSTM) et modèles à espace d’état (Mamba) pour la dynamique temporelle. Une attention particulière est portée à la comparaison entre modèles, aux horizons temporels d’impact (court, moyen, long terme) et à la robustesse des résultats. L’analyse empirique permet d’identifier les configurations dans lesquelles la FG contribue effectivement à réduire les tensions de marché, et celles où elle perd en efficacité, voire génère de l’instabilité. 
Enfin, ce dernier chapitre s’ouvre sur une discussion normative, qui plaide pour une intégration renforcée entre forward guidance et instruments de régulation macroprudentielle. Il est notamment suggéré que la communication monétaire gagne à être systématiquement adossée à des mesures concrètes, telles que la publication de calendriers d’intervention, l’activation de T-LTRO, ou la coordination avec les politiques de coussins contracycliques. Cette approche intégrée permettrait d’accroître la crédibilité de la guidance et de renforcer son effet de signal en période de tensions financières.\\

En définitive, ce mémoire ambitionne de mettre en lumière la puissance mais aussi les fragilités d’un outil devenu central dans les dispositifs de politique monétaire contemporaine.


