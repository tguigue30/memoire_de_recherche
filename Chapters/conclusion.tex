\chapter*{Conclusion}
\addcontentsline{toc}{chapter}{Conclusion}

Dans ce mémoire il a été démontré l’impact de la Forward Guidance de la Banque Centrale Européenne sur le stress systémique au sein de la zone euro mesuré par le CISS, en mobilisant des outils de NLP et d’apprentissage profond. Ainsi, dans ce mémoire, sur la base d’une approche textuelle, computationnelle et empirique, il a été montré dans quelle mesure la communication anticipative de la BCE permettait de réduire les tensions sur les marchés financiers, ou au contraire, pouvait les amplifier en période d’incertitude sur une période hebdomadaire allant de janvier 2005 à décembre 2024 sur données hebdomadaires.\\

Une analyse historique et conceptuelle a d'abord permis de retracer l’évolution du rôle de la forward guidance dans l’arsenal de la politique monétaire européenne. Apparue initialement comme une réponse à la borne inférieure des taux d’intérêt nominaux, la FG s’est progressivement institutionnalisée comme un instrument stratégique, mobilisé tant dans des contextes expansionnistes que restrictifs. Cependant, cette montée en puissance s’est accompagnée de nombreuses interrogations sur ses canaux de transmission, sa crédibilité, et son efficacité face aux chocs macro-financiers. Aussi, le concept de stress systémique s’est imposé comme une dimension centrale de la stabilité financière, incarnée notamment par la création de l’indicateur composite de stress systémique par la BCE. Contrairement aux simples mesures de volatilité, le stress systémique capte les dynamiques de contagion, les ruptures de confiance et les interdépendances entre marchés. Il constitue dès lors un outil pertinent pour étudier les effets potentiels de la communication monétaire sur l’architecture financière dans son ensemble.\\

Le mémoire a introduit une méthodologie innovante, articulée autour de l’analyse sémantique des discours de la Banque Centrale Européenne. Cette étape a consisté à mobiliser des modèles de langage de nouvelle génération, parmi lesquels ModernBERT, Mamba, Mamba-2 et Llamba, capables de capturer des représentations contextuelles fines et hiérarchisées des textes. Ces modèles, fondés sur des architectures de type transformeur ou espace d’état, ont permis d’extraire des vecteurs sémantiques à haute dimension, à partir desquels un indicateur directionnel de tonalité a été construit. Ce score continu, projeté dans un espace latent structuré, permet de quantifier la posture implicite de la BCE à chaque instant, sur un axe allant de l’orientation expansionniste à la posture plus restrictive. L’indicateur a été calculé pour l’ensemble des discours officiels de la BCE sur une période couvrant près de vingt ans (janvier 2005 à décembre 2024), intégrant ainsi une large diversité de contextes macroéconomiques et de régimes monétaires. Cette modélisation sémantique permet non seulement de saisir les nuances lexicales des communiqués, mais également de rendre compte des transitions progressives dans les discours, des inflexions de ton et des variations de cadre rhétorique (selon le président). Dans une dernière phase, cet indicateur a été intégré dans des modèles dynamiques de séries temporelles, afin d’en évaluer l’effet sur la trajectoire latente du stress systémique. À cette fin, des réseaux récurrents de type LSTM ont été mobilisés pour modéliser la dépendance séquentielle des données financières, tandis que l’espace latent a été régularisé par des techniques issues des Wasserstein Auto-Encoders. Cette approche a permis d’imposer une géométrie probabiliste cohérente à la représentation du stress, tout en autorisant une injection conditionnelle des signaux de forward guidance. L’ensemble du dispositif a ainsi permis d’évaluer, de manière intégrée, la capacité de la communication monétaire à influencer les dynamiques de vulnérabilité financière.\\

Les résultats empiriques mettent en évidence une hétérogénéité marquée dans l’impact de la FG selon l’horizon temporel et les régimes macro-financiers. À court terme, les résultats empiriques montrent que l’intégration de la FG dans les modèles conditionnels n’entraîne qu’une amélioration marginale des erreurs de reconstruction du stress systémique. Cette faible valeur ajoutée s’explique par la forte réactivité des marchés financiers, qui absorbent rapidement les informations publiques et ajustent leurs anticipations en conséquence. Néanmoins, la FG demeure un levier pertinent pour la stabilisation dans l'ancrage des agents économiques. Elle contribue à réduire l’asymétrie  entre acteurs, en clarifiant les intentions de la banque centrale, et permet d’aligner plus efficacement les décisions des participants de marché sur un cadre commun. Son utilité réside donc des les anticipations collectives, en particulier en période de transition ou d’incertitude modérée. À l’horizon moyen terme (environ 30 jours), les effets de la FG deviennent plus ambigus. En contexte de stabilité macro-financière, la tonalité des discours conserve une valeur explicative modérée : elle continue d’influencer la dynamique latente du stress en guidant progressivement les réajustements de portefeuille et les anticipations de liquidité. Toutefois, dans les épisodes de crise ou d’incertitude aiguë, la cohérence des signaux émis par la BCE s’affaiblit. Le recours accru à des formulations conditionnelles, la multiplication des voix institutionnelles, et les différences de tons complexifient le traitement automatique du signal. Les performances du modèle conditionnel deviennent alors plus volatiles, et peuvent se dégrader par rapport au modèle non conditionné, traduisant un bruit accru. La FG peut, dans ces conditions, introduire une incertitude supplémentaire, en altérant la lisibilité globale des marchés. À long terme, les résultats suggèrent que l’introduction de la Forward Guidance dans le modèle conditionnel n'améliore pas la capacité à anticiper les dynamiques latentes du stress systémique. Bien que les erreurs moyennes soient plus faibles en moyenne pour le modèle conditionnel, une observation plus fine des trajectoires révèle un retard persistant dans la dynamique reconstruite. La courbe du modèle conditionnel apparaît régulièrement décalée vers l’aval, indiquant que l’information contenue dans les discours de la BCE est intégrée avec un temps de latence (environ 6 mois), et que la Forward Guidance agit davantage comme un révélateur a posteriori que comme un signal prédictif immédiat. Ce décalage temporel peut s’interpréter comme le reflet d’un processus lent de transmission des signaux discursifs vers les anticipations effectives des agents économiques. Contrairement aux instruments directs de politique monétaire, la Forward Guidance ne produit pas d’effet instantané : sa réception, son interprétation et son incorporation dans les arbitrages financiers nécessitent un temps d’ajustement. Les comportements d’investissement, de couverture ou de refinancement peuvent ne répondre que progressivement aux inflexions de langage, surtout lorsque le discours reste ambigu, conditionnel ou sujet à interprétation. Ainsi, la tonalité agrégée des discours ne constitue pas ici un indicateur avancé du stress, mais plutôt un signal coïncident voire légèrement retardé dans les phases de tensions systémiques. Cela ne remet pas en cause la pertinence de la FG comme outil stratégique, mais souligne la nécessité de mieux encadrer ses modalités de diffusion et de renforcer sa lisibilité. Un discours plus clair, mieux synchronisé avec les autres instruments de politique économique, pourrait réduire ce décalage et renforcer l’efficacité temporelle de la communication anticipative.

\newpage

Il apparaît que la FG ne peut être traitée comme un instrument isolé : son efficacité dépend de la complémentarité avec d’autres leviers de politique monétaire et macroprudentielle. En période d’incertitude accrue, la simple émission de signaux verbaux, non assortis de mesures concrètes, tend à perdre de son impact. La coordination inter-institutionnelle (banques centrales, régulateurs prudentiels, autorités budgétaires) devient alors essentielle pour renforcer la crédibilité du signal. Ensuite, la stabilité des anticipations dépend fortement de la clarté sémantique et de la constance temporelle des messages. Une forward guidance changeante, ambiguë ou après coup modifiée peut engendrer de l’instabilité interprétative, en particulier dans des contextes de fragmentation des marchés ou de divergence des anticipations nationales. D’où l’importance d’un ancrage linguistique cohérent. En outre, les résultats empiriques de ce mémoire confirment que les outils issus de l’intelligence artificielle (IA) peuvent jouer un rôle croissant dans l’analyse de la politique monétaire. En capturant des dimensions sémantiques fines, ces modèles permettent de construire des indicateurs dynamiques sensibles aux évolutions discursives, et utiles à la fois pour l’analyse académique et pour le suivi opérationnel par les autorités.\\

Dans cette perspective, plusieurs enseignements opérationnels peuvent être formulés à destination des régulateurs. D’abord, les résultats plaident en faveur d’une structuration plus rigoureuse de la communication monétaire. Lorsque les messages sont multiples, fragmentés ou conditionnels, leur valeur informative s’amenuise. Il serait pertinent que les autorités adoptent des lignes directrices plus explicites en matière de tonalité et de constance des discours, en particulier en période de crise, où les risques d’interprétation divergente sont amplifiés. Cela implique de mieux coordonner les prises de parole entre les membres des comités de politique monétaire et de limiter les écarts narratifs susceptibles de brouiller la lecture des signaux par les agents économiques. Ensuite, il apparaît nécessaire de renforcer le lien  de la Forward Guidance à des actions mesurables. Les annonces doivent être accompagnées de points observables, comme des plages de taux, des seuils macroéconomiques ou des éléments calendaires, afin de maximiser leur pouvoir d’ancrage. La parole seule ne suffit que si elle s’inscrit dans un continuum stratégique crédible. À cet égard, la guidance devrait être systématiquement articulée avec les outils de liquidité pour éviter tout écart entre les intentions déclarées et les instruments effectivement mobilisés.\\

Il serait également recommandé d’intégrer la FG dans un système de suivi sémantique en temps réel. Les banques centrales pourraient tirer parti des techniques de traitement du langage développées ici pour surveiller l’évolution de la perception externe de leur communication. Des tableaux de bord internes, fondés sur des modèles de langage modernes, pourraient permettre d’évaluer en continu la stabilité, la dispersion et la direction des signaux envoyés. En cas de divergence marquée ou d’instabilité dans la tonalité perçue, des ajustements correctifs pourraient être envisagés rapidement. Ce qui permettrait de renforcer l’agilité de la politique monétaire dans un environnement actuel pris dans les crises. Enfin, la temporalité différée de l’effet de la Forward Guidance appelle à une révision des horizons de pilotage des régulateurs. Il ne suffit pas de mesurer l’impact immédiat d’une communication sur les marchés pour en juger la pertinence. Il convient de suivre son influence sur les anticipations et la structure des bilans bancaires à moyen et long termes. Cette approche multi-horizons permettrait d’éviter une focalisation excessive sur les réactions de court terme, souvent bruitées, au profit d’une évaluation plus différée.\\

Cependant, plusieurs limites demeurent et ouvrent la voie à de nouvelles pistes de recherche. L’analyse a été centrée sur les discours institutionnels de la BCE. Il serait intéressant d’y intégrer les discours décentralisés ou encore les échos médiatiques, pour cartographier la diffusion effective des signaux de forward guidance. De plus, les modèles de langage utilisés reposent sur des bases généralistes : l'entraînement sur des corpus spécifiquement monétaires pourrait améliorer leur sensibilité aux nuances techniques et macroéconomiques. Enfin, une ouverture majeure réside dans l’analyse de l’espace latent généré par les Wasserstein Auto-Encoders. Si ceux-ci permettent d’inférer des représentations structurellement régularisées du stress systémique, l’espace latent résultant pourrait faire l’objet d’une exploration plus approfondie. Il serait notamment prometteur de développer des méthodes capables de détecter les points de retournement structurels dans cet espace c’est-à-dire des configurations latentes préfigurant un changement de régime, une contagion accrue ou un basculement de la dynamique systémique. À terme, une formalisation géométrique plus riche de cet espace latent pourrait être envisagée. En s'appuyant sur des hypothèses de courbure issues de la géométrie riemannienne, par exemple via l’utilisation de variétés à courbure négative, il serait alors possible de structurer l’espace latent de manière à mieux refléter les transitions non linéaires. Cette perspective ouvre un champ de recherche à la croisée de la géométrie différentielle, du deep learning et de l’analyse de la politique monétaire. Dans un monde financier interconnecté, comprendre la morphologie des signaux latents et leur évolution pourrait fournir un avantage décisif pour anticiper les vulnérabilités systémiques. En somme, l’intelligence artificielle ne constitue pas seulement un outil d’analyse, mais bien un nouvel outil pour la stabilité systémique.\\

Enfin, une ouverture complémentaire peut être formulée autour de l’étude des mécanismes de contagion financière et macroéconomique, en particulier au sein du système interbancaire européen. Si le stress systémique a été ici modélisé sous la forme d’une trajectoire latente globale, reflétant les tensions agrégées sur les principaux marchés financiers, il demeure possible d’affiner cette lecture en analysant la manière dont ce stress se diffuse au sein du réseau bancaire. Le système interbancaire constitue en effet un vecteur privilégié de transmission des chocs de liquidité, des ruptures de confiance et des ajustements de portefeuilles induits par les signaux de politique monétaire. L’architecture WAE-LSTM développée dans ce mémoire pourrait ainsi être adaptée pour intégrer explicitement des données de réseau bancaire, telles que les expositions croisées. Une telle extension permettrait d’identifier si, et comment, les signaux de Forward Guidance contribuent à amplifier ou atténuer les dynamiques de contagion entre banques. Par exemple, une communication perçue comme restrictive pourrait accroître les tensions de liquidité en périphérie du réseau, provoquant une fragmentation accrue du marché interbancaire. Inversement, une guidance perçue comme crédible et coordonnée pourrait renforcer la confiance mutuelle et réduire les besoins de précaution au bilan, limitant ainsi la propagation du stress. L’analyse des effets de la FG dans l’architecture financière pourrait ainsi devenir un axe à part entière. Plus largement, cette perspective invite à repenser la politique monétaire comme un processus d’influence systémique multi-échelle. Au-delà de son impact agrégé sur les anticipations de marché, la Forward Guidance produit des effets différenciés selon la position institutionnelle des agents et leur rôle dans la chaîne de transmission financière. Intégrer cette nuance  permettrait non seulement d’enrichir les modèles prédictifs du stress, mais aussi d’améliorer la capacité des régulateurs à anticiper les points de rupture. D'autre part, à cibler leurs interventions selon les agents.

\newpage

Ce mémoire a donc contribué à une réflexion plus large sur l’avenir de la politique monétaire dans un environnement numérique ett saturé d’informations. Dans ce contexte, la forward guidance pourrait évoluer vers une communication plus interactive et stratégiquement augmentée par les technologies d’IA. Il est envisageable que les banques centrales disposent, à l’avenir, de tableaux de bord en temps réel intégrant des indicateurs de texte et des alertes. Au-delà de la seule stabilité financière, cette approche ouvre la voie à une politique monétaire mieux informée des dynamiques comportementales et donc plus réactive aux signaux de désancrage des anticipations. La communication deviendrait ainsi un important poids pour la résilience systémique. En somme, l’intégration des méthodes d’IA dans l’analyse monétaire ne vise pas à remplacer le jugement du banquier central, mais à en renforcer la lucidité, à condition de rester rigoureusement encadrée et de respecter les principes de transparence. Ces approches pourraient constituer une avancée dans la conduite de la politique monétaire future.