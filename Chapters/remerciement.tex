\phantomsection
\addcontentsline{toc}{chapter}{Remerciements}

\vspace{3em}

\begin{center}
  {\Huge\bfseries Remerciements}
\end{center}

\vspace{1.5em}

\begin{adjustwidth}{2em}{2em}
\begin{sloppypar}

Je tiens à exprimer ma profonde gratitude à Madame \textbf{Françoise Seyte}, pour son accompagnement rigoureux, sa disponibilité constante et la confiance qu’elle m’a accordée tout au long de ce travail en m'y laissant une liberté totale. Sa bienveillance, son exigence méthodologique et la richesse de ses retours ont été déterminantes dans l’élaboration de ce mémoire. Je lui suis infiniment reconnaissant pour l’ensemble de ses conseils, son soutien et l’inspiration qu’elle m’a apportée à chaque étape de cette recherche.\\

Je remercie également Monsieur \textbf{Stéphane Mussard} pour ses conseils éclairés sur les aspects méthodologiques du mémoire, en particulier en ce qui concerne l’apprentissage profond et la modélisation des séries temporelles. Ses suggestions ont grandement enrichi la dimension technique de ce travail, tout en m’incitant à pousser plus loin la rigueur et la précision dans les choix de modélisation.\\

Enfin, je souhaite remercier très chaleureusement Monsieur \textbf{Roman Mestre}, pour son rôle en tant que co-encadrant. Son expertise en macroéconomie monétaire ainsi que ses explications ont contribué à ancrer ce mémoire dans une perspective économique solide et cohérente. Ses conseils ont été précieux pour articuler les enjeux théoriques et empiriques de ce travail.\\

À chacun d’eux, je tiens à témoigner toute ma reconnaissance pour leur accompagnement exigeant, stimulant et toujours bienveillant.
\end{sloppypar}
\end{adjustwidth}
